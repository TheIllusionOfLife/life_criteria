\documentclass[letterpaper]{article}
\usepackage{natbib,alifeconf}
\usepackage{graphicx,amsmath,booktabs,xcolor,multirow}

\title{Digital Life: Satisfying Seven Biological Criteria\\Through Functional Analogy and Criterion-Ablation}
\author{Anonymous}

\begin{document}
\maketitle

% ============================================================================
\begin{abstract}
We present the first artificial life system that integrates all seven textbook
biological criteria for life---cellular organization, metabolism, homeostasis,
growth, reproduction, response to stimuli, and evolution---as functionally
interdependent computational processes.
Existing systems implement at most a subset of these criteria, often as
independent modules or static proxies that can be removed without measurable
system degradation.
Our hybrid swarm-organism architecture implements each criterion as a dynamic
process satisfying three conditions---sustained resource consumption,
measurable degradation upon removal, and feedback coupling with other
criteria---which we term \emph{functional analogy}.
A criterion-ablation experiment ($n$=30 per condition, seeds held out from
calibration) demonstrates that disabling any single criterion causes
statistically significant population decline (Mann-Whitney $U$,
Holm-Bonferroni corrected, all $p \leq 0.016$), with Cohen's $d$ ranging from
0.57 to 18.1.
The strongest effects arise from disabling reproduction ($\Delta$=--89.3\%),
response to stimuli ($\Delta$=--88.4\%), and metabolism ($\Delta$=--84.3\%),
confirming that these criteria function as necessary, interdependent
components of organismal viability rather than decorative labels.
\end{abstract}

% ============================================================================
\section{Introduction}

What distinguishes a living system from a merely complex one?
Biology textbooks identify seven criteria---cellular organization, metabolism,
homeostasis, growth and development, reproduction, response to stimuli, and
evolution \citep{urry_2020_campbell}---but artificial life (ALife) research has
struggled to integrate all seven into a single computational system.
Most existing platforms implement a subset: evolutionary dynamics without
metabolism \citep{ofria_2004_avida}, pattern formation without reproduction
\citep{chan_2019_lenia}, or boundary self-organization without evolution
\citep{plantec_2023_flow}.
Where criteria are nominally present, they often function as \emph{simplified
proxies}---static parameters or independent modules whose removal has no
measurable effect on system behavior.

We argue that a meaningful computational model of life requires more than
feature checklists.
Each criterion must function as a \emph{functional analogy} of its biological
counterpart, satisfying three conditions:
\begin{enumerate}
  \item \textbf{Dynamic process}: the criterion requires sustained resource
    consumption at every timestep, not a static lookup.
  \item \textbf{Measurable degradation}: ablating the criterion causes
    statistically significant decline in organism viability.
  \item \textbf{Feedback coupling}: the criterion forms at least one feedback
    loop with another criterion, precluding independent-module implementations.
\end{enumerate}
This definition operationalizes the intuition behind autopoiesis
\citep{maturana_1980_autopoiesis} and minimal-life frameworks
\citep{ruizmirazo_2004_universal} in a form amenable to experimental
falsification.

We adopt a \emph{weak ALife} stance: our system is a functional model of life,
not a claim that the organisms are alive.
The contribution is methodological---demonstrating that all seven criteria
\emph{can} be integrated as interdependent processes and that their necessity
can be rigorously tested.

This paper makes three contributions:
\begin{enumerate}
  \item A hybrid swarm-organism architecture integrating all seven biological
    criteria as functionally interdependent processes.
  \item An operational definition of \emph{functional analogy} with three
    falsifiable conditions.
  \item A \emph{criterion-ablation} methodology proving each criterion's
    functional necessity via controlled experiments with multiple-comparison
    correction.
\end{enumerate}

% ============================================================================
\section{Related Work}

Artificial life research has produced diverse computational substrates, each
excelling along different axes of biological fidelity.
We organize the landscape by methodological approach rather than chronology.

\paragraph{Evolutionary platforms.}
Tierra \citep{ray_1991_approach} and Avida \citep{ofria_2004_avida} pioneered
self-replicating digital organisms with mutation and natural selection,
achieving strong evolutionary dynamics (Level~5 on our rubric).
However, organisms lack spatial bodies, metabolic processes, and boundary
maintenance---criteria 1--4 receive minimal or no implementation.
Polyworld \citep{yaeger_1994_computational} adds neural-network-driven behavior
and simple energy budgets, though its metabolism operates as a dynamic
single-resource process rather than a multi-step network.

\paragraph{Continuous and particle-based systems.}
Lenia \citep{chan_2019_lenia} demonstrates emergent lifelike patterns in
continuous cellular automata, exhibiting coherent spatial organization and
growth.
Flow-Lenia \citep{plantec_2023_flow} extends this with mass conservation,
achieving stronger boundary maintenance.
ALIEN \citep{heinemann_2008_alien} provides a GPU-accelerated particle
simulator with typed cells supporting self-replicating structures,
achieving multi-process interaction on several criteria.
While ALIEN demonstrates broad coverage, no existing system combines
multi-step metabolism with active homeostatic regulation.

\paragraph{Chemistry-inspired systems.}
Coralai \citep{barbieux_2024_coralai} combines multi-agent neural cellular
automata with energy dynamics, implementing rudimentary metabolism and
spatial organization.
It represents a step toward metabolic integration but does not achieve
the feedback coupling between metabolism and other criteria that our
framework requires.

\paragraph{Theoretical foundations.}
The autopoiesis framework \citep{maturana_1980_autopoiesis,
mcmullin_2004_thirty} emphasizes self-producing boundaries as the
minimal criterion for life, which our boundary-maintenance mechanism
operationalizes.
NASA's working definition---``a self-sustaining chemical system capable of
Darwinian evolution'' \citep{joyce_1994_foreword}---bundles metabolism and
evolution but does not individuate the remaining five criteria.
\citet{ruizmirazo_2004_universal} argue for a richer set of minimal
conditions, closer to our seven-criteria framework.
The open-ended evolution research program \citep{bedau_2000_open,
taylor_2016_openended} provides metrics for evolutionary richness that complement
our criterion-ablation approach.

Table~\ref{tab:comparison} summarizes how existing systems score on each
criterion using a five-level rubric (1=absent, 5=self-maintaining/emergent).

\begin{table*}[t]
\centering
\caption{Literature comparison: seven biological criteria scored on a
five-level rubric (1=no feature, 2=static parameter, 3=dynamic single process,
4=multi-process interaction, 5=self-maintaining/emergent).
Bold indicates scores $\geq$4.}
\label{tab:comparison}
\small
\begin{tabular}{l@{\hskip 6pt}c@{\hskip 6pt}c@{\hskip 6pt}c@{\hskip 6pt}c@{\hskip 6pt}c@{\hskip 6pt}c@{\hskip 6pt}c@{\hskip 6pt}c}
\toprule
System & Cell.Org & Metab & Homeo & Growth & Reprod & Response & Evol & Total \\
\midrule
% Scores verified via literature review; justifications in paper/literature_scores.md
Polyworld  & 2 & 3 & 1 & 1 & 3 & \textbf{4} & \textbf{4} & 18 \\
Avida      & 2 & 3 & 1 & 2 & \textbf{4} & 3 & \textbf{5} & 20 \\
Lenia      & 3 & 1 & 2 & 2 & 2 & 3 & 2 & 15 \\
ALIEN      & \textbf{4} & 3 & 2 & 3 & \textbf{4} & \textbf{4} & \textbf{4} & 24 \\
Flow-Lenia & 3 & 3 & 3 & 3 & 3 & 3 & 3 & 21 \\
Coralai    & 3 & 3 & 2 & 3 & 3 & 3 & 3 & 20 \\
\midrule
\textbf{Ours} & \textbf{4} & \textbf{5} & \textbf{4} & 3 & \textbf{4} & \textbf{4} & 3 & \textbf{27} \\
\bottomrule
\end{tabular}
\end{table*}

While ALIEN achieves Level~4 on four criteria, it lacks multi-step metabolism
(Level~3) and active homeostasis (Level~2).
Our system achieves Level~5 on metabolism---the only system to do so---and
reaches $\geq$4 on five of seven criteria, with the highest total score (27)
across all systems surveyed.

% ============================================================================
\section{System Design}

\subsection{Architecture Overview}

The system implements a hybrid two-layer architecture (Figure~\ref{fig:arch}).
The outer layer is a continuous toroidal 2D environment (100$\times$100 world
units) containing a diffusing resource field.
The inner layer consists of 10--50 \emph{organisms}, each composed of 10--50
\emph{swarm agents} that collectively maintain the organism's spatial boundary.

\begin{figure*}[t]
\centering
\includegraphics[width=\textwidth]{figures/fig_architecture.pdf}
\caption{Two-layer architecture. Each organism comprises swarm agents maintaining
a spatial boundary, a neural-network controller, a graph-based metabolic network,
and a variable-length genome encoding all seven criteria. Organisms inhabit a
continuous toroidal environment with a diffusing resource field.}
\label{fig:arch}
\end{figure*}

Each organism maintains the following runtime state: boundary integrity
($b \in [0,1]$), metabolic state (energy $e$, resource $r$, waste $w$),
internal state vector for homeostatic regulation, a neural-network controller,
a genetically encoded metabolic network, age, generation counter, and maturity
level.

\subsection{Seven Criteria Implementation}

Table~\ref{tab:criteria} maps each biological criterion to its computational
implementation, ablation toggle, and feedback partners.

\begin{table}[t]
\centering
\caption{Mapping of seven biological criteria to computational processes.
Each criterion satisfies the three functional-analogy conditions.}
\label{tab:criteria}
\small
\begin{tabular}{@{}p{1.4cm}p{2.8cm}p{2.5cm}@{}}
\toprule
Criterion & Process & Feedback \\
\midrule
Cell.\ Org. & Swarm agents maintain boundary; decays without energy & Metab, Homeo \\
Metabolism & Graph network transforms resources to energy; waste accumulates & Cell.\ Org., Homeo \\
Homeostasis & NN regulates internal state vector each step & Metab, Response \\
Growth & Maturation from seed to full capacity & Metab, Reprod \\
Reproduction & Division when metabolically ready; energy cost; offspring from seed & Metab, Evol \\
Response & NN processes sensory input $\rightarrow$ velocity delta & Homeo, Metab \\
Evolution & Mutation during reproduction; differential survival & Reprod, all \\
\bottomrule
\end{tabular}
\end{table}

\paragraph{Cellular organization.}
Swarm agents collectively define an organism's spatial extent.
Boundary integrity $b$ decays each step at a base rate modulated by energy
deficit and waste pressure:
$\Delta b_{\text{decay}} = -r_b \cdot (1 + s_e \cdot (1 - e) + s_w \cdot w)$,
where $r_b = 0.02$ is the base decay rate, $s_e = 0.5$ and $s_w = 0.3$ are
scaling factors.
Repair occurs proportionally to available energy:
$\Delta b_{\text{repair}} = r_r \cdot e \cdot (1 - s_p \cdot w)$,
with repair rate $r_r = 0.15$ and waste penalty $s_p = 0.4$.
When $b$ falls below a collapse threshold ($b < 0.1$), the organism dies.

\paragraph{Metabolism.}
Each organism possesses a genetically encoded graph-based metabolic network
with 2--4 catalytic nodes and directed edges.
The genome segment (16 floats) is decoded via sigmoid mapping:
node count $= \text{round}(\sigma(g_0) \cdot 2 + 2)$,
catalytic efficiency $= \sigma(g_{2+i}) \cdot 0.9 + 0.1 \in [0.1, 1.0]$,
edge existence determined by $|g_j| > 0.3$, and
conversion efficiency $= \sigma(g_{13}) \cdot 0.7 + 0.3 \in [0.3, 1.0]$.
External resources enter at a designated entry node, flow through the graph
with per-edge transfer efficiency in $[0.7, 1.0]$, and exit as energy.
Waste accumulates as a byproduct proportional to throughput.

\paragraph{Homeostasis.}
A feedforward neural network (8 inputs $\rightarrow$ 16 hidden with tanh
$\rightarrow$ 4 outputs with tanh; 212 weights) processes sensory inputs
(position, velocity, internal state, neighbor count) and produces velocity
adjustments and internal-state deltas.
The internal state vector enables adaptive regulation: organisms that maintain
internal variables within viable ranges survive longer.

\paragraph{Growth and development.}
Organisms begin as minimal seeds (maturity $m = 0$) and develop toward full
capacity ($m = 1$) over time.
Maturation gates metabolic throughput and reproductive readiness, ensuring
organisms must develop before they can reproduce.

\paragraph{Reproduction.}
When energy exceeds $e_{\min} = 0.7$ and boundary integrity exceeds
$b_{\min} = 0.5$, an organism may divide.
The parent pays an energy cost ($c_r = 0.3$), and the offspring inherits a
(possibly mutated) copy of the genome, starting as a seed.
Child agents spawn within a radius of the parent's center of mass.

\paragraph{Response to stimuli.}
The neural-network controller processes a local sensory field each timestep,
producing velocity deltas that govern agent movement.
Disabling response freezes agents' velocity adjustments, preventing adaptive
resource seeking.

\paragraph{Evolution.}
During reproduction, offspring genomes undergo point mutations
(rate $= 0.01$ per gene, scale $= 0.1$), reset mutations
(rate $= 0.001$), and scale mutations (rate $= 0.005$,
factor $\in [0.8, 1.2]$).
All gene values are clamped to $[-5, 5]$.
This produces heritable variation subject to differential survival.

\subsection{Genome Encoding}

The genome is a variable-length vector of 256 floats organized into seven
segments: neural-network weights (212), metabolic network (16), homeostasis
parameters (8), developmental program (8), reproduction parameters (4),
sensory parameters (4), and evolution parameters (4).
All criteria are encoded from initialization; segments are activated as
features are enabled.

% ============================================================================
\section{Criterion-Ablation Experiment}

This experiment tests whether each of the seven criteria is functionally
necessary for organism viability, as predicted by the functional-analogy
framework.

\subsection{Protocol}

The system provides seven boolean ablation toggles, one per criterion
(e.g., \texttt{enable\_metabolism = false}).
For each of the seven criteria, we disable that criterion while keeping all
others active, and compare the resulting population dynamics against the
fully enabled baseline (``normal'' condition).
This yields eight conditions: one normal baseline and seven single-criterion
ablations.

\subsection{Data Separation}

To prevent overfitting of thresholds, we separate data into:
\begin{itemize}
  \item \textbf{Calibration set}: Seeds 0--99, used during development for
    parameter tuning and threshold selection.
  \item \textbf{Test set}: Seeds 100--129 ($n$=30), held out until final
    evaluation. All reported results use this set exclusively.
\end{itemize}

Calibration confirmed that both metabolism engines produce viable populations
(Toy: $\bar{x}$=328.1, Graph: $\bar{x}$=291.8 alive at step 2000).
All final experiments use the Graph metabolism engine.

\subsection{Simulation Parameters}

Each simulation runs for 2000 timesteps with population sampled every 50
steps.
The environment is a 100$\times$100 toroidal grid with 30 initial organisms,
each comprising 25 swarm agents.
The primary outcome metric is alive organism count at step 2000.

\subsection{Statistical Design}

For each ablation condition, we test the one-sided hypothesis:
\[
H_1: \text{alive\_count}_{\text{normal}} > \text{alive\_count}_{\text{ablated}}
\]
using the Mann-Whitney $U$ test \citep{mann_1947_test}, appropriate for
non-normal count data.
We apply Holm-Bonferroni correction \citep{holm_1979_simple} for seven
simultaneous comparisons at $\alpha = 0.05$.
Effect sizes are reported as Cohen's $d$ \citep{cohen_1988_statistical}.

% ============================================================================
\section{Results}

All seven criterion ablations produce statistically significant population
decline compared to the normal baseline (Table~\ref{tab:ablation}).

\begin{table}[t]
\centering
\caption{Criterion-ablation results ($n$=30 per condition). Normal baseline
mean: 293.1. All $p$-values Holm-Bonferroni corrected. $^{***}p<0.001$,
$^{*}p<0.05$.}
\label{tab:ablation}
\small
\begin{tabular}{@{}lrrrrr@{}}
\toprule
Condition & Mean & $\Delta$\% & $d$ & $p_{\text{corr}}$ & Sig. \\
\midrule
No Reproduction & 31.3 & $-$89.3 & \textbf{18.06} & $<$0.001 & $^{***}$ \\
No Response     & 34.1 & $-$88.4 & \textbf{17.82} & $<$0.001 & $^{***}$ \\
No Metabolism   & 46.0 & $-$84.3 & \textbf{17.24} & $<$0.001 & $^{***}$ \\
No Homeostasis  & 68.3 & $-$76.7 & 4.94 & $<$0.001 & $^{***}$ \\
No Growth       & 185.6 & $-$36.7 & 5.34 & $<$0.001 & $^{***}$ \\
No Boundary     & 121.8 & $-$58.4 & 4.36 & $<$0.001 & $^{***}$ \\
No Evolution    & 278.3 & $-$5.1  & 0.57 & 0.016 & $^{*}$ \\
\bottomrule
\end{tabular}
\end{table}

Three ablations cause near-total population collapse ($>$84\% decline):
reproduction, response to stimuli, and metabolism.
These criteria form the core viability loop---without energy production,
adaptive movement, or population renewal, organisms cannot sustain themselves.

Figure~\ref{fig:timeseries} shows population trajectories across all
conditions.
The normal condition (black) stabilizes around 293 organisms by step~1000.
Metabolic ablation (orange) causes rapid collapse within the first 200 steps,
as organisms cannot produce energy to maintain boundaries.
Reproduction ablation (blue) produces a slower but equally terminal decline,
as the initial population ages and dies without replacement.
Evolution ablation (purple) shows the weakest effect ($d$=0.57), with
populations remaining viable but slightly smaller than normal---consistent
with evolution operating as an optimization process rather than a survival
necessity at these timescales.

\begin{figure*}[t]
\centering
\includegraphics[width=\textwidth]{figures/fig_timeseries.pdf}
\caption{Population dynamics under criterion ablation. Lines show mean alive
count across 30 seeds (100--129); shaded bands show $\pm$1 SEM. Normal
baseline (thick black) stabilizes near 293 organisms. Removing reproduction,
response, or metabolism causes $>$84\% population collapse. Evolution ablation
shows a modest 5\% decline ($d$=0.57), consistent with optimization rather
than short-term survival necessity.}
\label{fig:timeseries}
\end{figure*}

\paragraph{Functional analogy verification.}
For each criterion, all three conditions are satisfied:
(a)~each consumes resources per step (energy for boundary repair, metabolic
computation, NN evaluation);
(b)~ablation causes significant degradation (Table~\ref{tab:ablation}); and
(c)~feedback loops are observable (e.g., metabolism~$\leftrightarrow$~boundary:
energy funds repair, boundary collapse stops metabolism).
Thus, each criterion qualifies as a functional analogy, not a simplified proxy.

% ============================================================================
\section{Discussion}

\paragraph{Criterion interdependence.}
The ablation results reveal a hierarchy of necessity.
Reproduction, response, and metabolism form an essential triad---their removal
causes $>$84\% population collapse.
Homeostasis and boundary maintenance occupy a middle tier ($\sim$58--77\%
decline), modulating organism fitness without being immediately lethal.
Growth occupies a similar middle tier ($\sim$37\% decline).
Evolution shows the weakest short-term effect ($\sim$5\% decline), suggesting
that at 2000-step timescales, evolutionary optimization contributes marginal
improvement rather than constituting a survival requirement.
This is biologically plausible: individual organisms do not need evolution to
survive, but populations benefit from adaptation over longer horizons.

\paragraph{Graph metabolism as functional analogy.}
The graph-based metabolic network---genetically encoded, multi-node, with
per-edge transfer efficiencies---represents a genuine multi-step
transformation process rather than a simple energy counter.
Its removal causes the third-strongest ablation effect ($d$=17.24),
confirming that metabolic complexity contributes to viability beyond what a
static energy parameter would provide.

\subsection{Limitations}

Several limitations constrain interpretation:

\textbf{Growth mechanism.}
The current growth/development implementation uses a maturation toggle rather
than a full developmental program.
While ablation still produces significant degradation ($d$=5.34), a richer
developmental model (e.g., morphogenetic agent recruitment) would strengthen
the functional-analogy claim for this criterion.

\textbf{Evolution timescale.}
The modest evolution effect ($d$=0.57) reflects the 2000-step simulation
horizon.
Evolutionary dynamics operate across generations; demonstrating stronger
evolutionary necessity would require runs of $10^4$--$10^5$ steps with
environmental change to create sustained selection pressure.

\textbf{Computational scale.}
The system runs on a single Mac Mini M2 Pro, limiting population sizes to
$\sim$300 organisms.
Larger populations might reveal emergent phenomena (e.g., speciation,
ecosystem dynamics) not observable at current scale.

\textbf{Weak ALife framing.}
We make no claim that these digital organisms are alive.
The functional-analogy framework demonstrates that seven criteria \emph{can}
be integrated as interdependent processes, but this remains a model, not
a sufficient condition for life.

% ============================================================================
\section{Conclusion}

We presented the first artificial life system integrating all seven textbook
biological criteria as functionally interdependent processes, verified through
controlled criterion-ablation experiments.
The functional-analogy framework---requiring dynamic operation, measurable
degradation upon removal, and feedback coupling---provides a rigorous
standard distinguishing genuine criteria implementations from simplified
proxies.

Our results demonstrate that no criterion is decorative: removing any one
causes statistically significant population decline ($p < 0.016$, Holm-Bonferroni
corrected), with effect sizes spanning from modest evolutionary optimization
($d$=0.57) to catastrophic metabolic failure ($d$=17.24).

Future work will pursue three directions:
(1)~longer simulation runs ($10^4$+ steps) to reveal stronger evolutionary
dynamics and open-ended evolution metrics \citep{bedau_2000_open,
taylor_2016_openended};
(2)~a richer developmental program replacing the current growth toggle; and
(3)~scaling to larger populations to investigate emergent ecological
phenomena.

\bibliographystyle{apalike}
\bibliography{references}

\end{document}
